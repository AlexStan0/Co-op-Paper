
% !BIB TS-program = biber
\documentclass[a4paper, 12pt, oneside]{article}

\usepackage[utf8]{inputenc} 
\usepackage[T1]{fontenc} 
\usepackage{fouriernc} 
\usepackage[margin=1in]{geometry}
\usepackage{fancyhdr}
\usepackage{tocloft}

% Configure page style to be "fancy"
\fancyhf{}
\pagestyle{fancy}

% Make the header fill the page width
\addtolength{\headwidth}{\marginparsep}
\addtolength{\headwidth}{\marginparwidth}

% Set the footer and header widths to be the same
\renewcommand{\footrulewidth}{1pt}
\renewcommand{\headrulewidth}{1pt}

\fancyhead[R]{Stan \thepage}

% Center the heading for the ToC
\renewcommand{\cfttoctitlefont}{\hfill\huge\bfseries}
\renewcommand{\cftaftertoctitle}{\hfill}

% Set the name of the table of contents
\renewcommand{\contentsname}{Table of Contents}

\begin{document} 

    \begin{titlepage} 

        \centering 
        
        \scshape 
        
        \vspace*{\baselineskip} 
        
        %------------------------------------------------
        %	Title
        %------------------------------------------------
        
        \rule{\textwidth}{1.6pt}\vspace*{-\baselineskip}\vspace*{2pt} % Thick horizontal rule
        \rule{\textwidth}{0.4pt} % Thin horizontal rule
        
        \vspace{0.75\baselineskip} 
        
        {\LARGE HUMAN MOTION PERCEPTION \\ IN VIRTUAL REALITY AND THE REAL-WORLD} 
        
        \vspace{0.75\baselineskip} 
        
        \rule{\textwidth}{0.4pt}\vspace*{-\baselineskip}\vspace{3.2pt} 
        \rule{\textwidth}{1.6pt} 
        
        \vspace{2\baselineskip}
        
        %------------------------------------------------
        %	Subtitle
        %------------------------------------------------
        
        Comparing Human Motion Perception in Virtual Reality 
        and Real-World Simultaneous Motion
        
        \vspace*{3\baselineskip}
        
        \vspace{0.5\baselineskip} 
        
        {\scshape\Large Alexandru Stan } 
        
        \vspace{0.5\baselineskip} 
        
        \textit{The University of Waterloo} 
        
        \vfill 

        SUPERVISED BY DR. WANG\\
        2024 % Publication year

    \end{titlepage}

    \setlength{\parskip}{2ex}
    \tableofcontents
    \clearpage

    \section{Abstract}
    
        This study delves into the nuanced realm of the Just Noticable Difference (JND) in 
        human motion perception, specifically examining the distinctions betwen physical reality 
        and virtual reality (VR) environements. Employing a HP Reverb G2 VR headset coupled with a 
        custom-designed go-kart featuring a swerve drivebase and bespoke components. 
        \newline

        The core 
        objective was to discern the treshold at which participants could reliably perceive variances
        in motion between the real-world and the immersive VR setting, with the virtual speed and environements
        held as constants throughout the experiment. Participants underwent a diverse range of experiences, traversing through a spectrum of physcal 
        accelerations -- both incremental and decremental -- while seamlessly navigating a linear trajectory 
        within the virtual environement, courtesy of the go-kart. The experiment sought to unravel the intricacies
        of human perception, pinpointing the point at which individuals could conciously detect dispartities in motion
        between reality and VR. 
        \newline

        The results of the study unveil a sophisticated comprehension of the Just Noticable Difference (JND), shedding 
        illuminating inisights on the acuity of individuals in ditinguishing subtle variations in motion between the 
        tangible and virtual. This explorations into the interplaty of sensory cues not only advances the understanding of 
        human perception, but also has profound implications for the future of VR design and development. 
        In conclusion, this research contributes valuable knowledge to the evolving landscape of immersive technologies, 
        offering a deeper understanding of how individuals perceive and interact with motion is both real-world 
        and virtual scenarios. As the boundaris between the physical and virtual continue to blur, this results of 
        this study pave the way for future research and a better understanding on the human experience in this changing world. 
        \clearpage

    \section{Introduction}
        \subsection{Overview of VR Motion Perception Experiments}

            Currenly, Virtual Reality (VR) motion experiments the prevailing focus
            lies in post-VR session perception effects rather than the concurrent
            assessments of the motion perception in both real-life and virtual environements.
            Traditional studies delve into the intricacies of perception after a VR experience,
            often exploring the impact of cognitive mechanisms, such as in \textbf{The effects of virtual 
            reality, augmented reality, and motion parallax on egocentric depth perception} where 
            they is it observed that \"participants view the virtual world as compressed relative to the 
            the real world\" [Jones et al., 2008].

            The research emphasizes the role of top-down factors, such as the cognitive interpretation of the virtual environment, 
            in influencing motion perception. Naturalistic and coherent settings, like those depicting a 3D scene, are found to 
            enhance the believability of visual stimuli, contributing to more effective vection induction. This implies that the 
            visual stimulus's effectiveness in conveying self-motion is intricately tied to the perceived realism and ecological
             relevance of the virtual environment.

            Conversely, the underrealized potential of VR in the built environment sector points to a gap in understanding the 
            complex interaction between environmental design and users. While VR is widely used for design intent and construction 
            methodologies, there is a concern that its current utilization predominantly focuses on the visual modality, neglecting 
            the multi-sensory nature of spatial experiences. The need to address human perception issues in VR models for effective 
            communication in the built environment is underscored.

            Additionally, the evolving landscape of virtual and augmented reality applications demands a comprehensive understanding 
            of how observers perceive spatial relationships. Notably, the traditional emphasis on measuring egocentric depth 
            perception in VR has revealed systematic underestimations, suggesting a compression of virtual space relative to the 
            real world. Surprisingly, however, these effects may not be as pronounced in augmented reality. 


        \subsection{Significance of Studying Motion Perception in VR}

        Studying motion perception in Virtual Reality (VR) carries profound significance, with implications extending beyond 
        the confines of individual experiences to transformative shifts in our understanding of the built environment and the 
        intricacies of human cognition. Building upon existing research, which predominantly delves into post-VR session effects
         and cognitive mechanisms influencing motion perception, a comprehensive exploration of real-time motion perception in 
         both real-life and virtual environments becomes crucial.

        One notable significance lies in democratizing experiences that were previously constrained by cost and safety concerns. 
        For instance, VR facilitates the creation of immersive racetrack simulations, allowing individuals to virtually experience 
        high-speed racing without the logistical challenges or risks associated with real tracks. This democratization of 
        experiences can extend to various domains, offering people the opportunity to engage with environments and activities 
        that might be impractical, inaccessible, or hazardous in reality.
        
        Moreover, understanding motion perception in VR contributes to a deeper comprehension of the human mind's adaptability and 
        response to differences in movement. By closely studying how individuals perceive and navigate motion in real-time VR 
        scenarios, researchers gain insights into the intricacies of spatial cognition and sensory integration. This knowledge not 
        only aids in refining VR applications for optimal user experience but also sheds light on the human mind's ability to 
        reconcile disparities between virtual and physical stimuli.
        
        In the context of the built environment, where VR is increasingly utilized for conveying design intent and construction 
        methodologies, a nuanced understanding of motion perception allows for more effective communication tools. Beyond visual
         representations, incorporating realistic motion experiences enhances the authenticity of virtual environments. This not 
         only benefits professionals in architecture, construction, and design but also broadens the scope of VR applications in 
         areas where spatial understanding is pivotal.

        \subsection{My Role as a Research Assistant}

        As a research assistant, my primary responsibility centered around the creation of a dynamic and immersive virtual 
        environment tailored for the experimental needs of our study. Leveraging my skills in software development and 3D modeling, 
        I utilized Unity and Blender to craft a realistic and engaging digital space that served as the backdrop for participants 
        during the research experiment.

        In the initial stages, my role involved the meticulous design of 3D models representing various elements such as roads, 
        houses, cars, and other relevant features integral to the study. Employing Blender, I sculpted detailed and accurate 
        representations of these components to ensure an authentic virtual environment.

        Subsequently, I transitioned to Unity, where I orchestrated the integration of these 3D models into a cohesive virtual scene.
        This involved the creation of a custom terrain to mimic mountainous landscapes, providing participants with a seamless and 
        immersive experience. My role extended beyond mere placement, as I carefully considered the spatial relationships between 
        different elements to optimize the feel of the virtual environment.
        \clearpage

    \section{Methodology}
        \subsection{Instruments and Equipment Used}
    
        Participants in the experiment were seated in a custom go-kart made with unique parts and a swerve drivebase, offering precise 
        control over movements. This setup aimed to provide an authentic experience of motion in both real and virtual settings. The 
        swerve drivebase allowed for responsive interactions during the experiment.

        The virtual experience was facilitated using the HP Reverb G2 Virtual Reality headset, known for its display resolution of 
        2160x2160 pixels. This VR headset delivered a clear and immersive experience, enhancing the realism of the virtual environment. 
        Its features contributed to the overall experiment by seamlessly integrating physical and virtual sensations.

        All testing occurred in a spacious facility in Glasgow, ensuring a safe environment for go-kart maneuvering and participant 
        engagement with the virtual setting. This setting allowed for a straightforward exploration of motion perception, providing 
        valuable insights into human-machine interaction.

        \subsection{Variables and Controls}
        
        In this experimental study, the primary independent variables are the levels of physical acceleration applied to the custom 
        go-kart and the variations in the virtual environment presented through the HP Reverb G2 VR headset. These variables are 
        crucial for exploring how participants perceive motion in both the physical and virtual realms. By systematically 
        manipulating acceleration levels, researchers aim to understand the thresholds at which participants can discern differences 
        in motion, while variations in the virtual environment allow for the investigation of the impact of visual stimuli on motion 
        perception.

        To ensure the reliability and validity of the study, several controls are implemented. Baseline measurements are conducted 
        before the experiment to account for individual differences among participants. Consistency in VR hardware, specifically 
        the use of the HP Reverb G2 headset, is maintained across all participants to eliminate potential variations in the virtual 
        experience. The uniform configuration of the custom go-kart serves as another control, ensuring that the physical setup 
        remains consistent for all participants. The testing environment itself, a spacious facility in Glasgow, is controlled to 
        minimize external influences and provide a standardized setting for participants.

        Throughout the study, ethical considerations remain a paramount control. Informed consent is a must, 
        and their safety is prioritized, particularly given the physical nature of the custom go-kart. By carefully manipulating 
        independent variables and implementing controls, this research aims to provide valuable insights into the complex interplay 
        between physical and virtual motion perception, contributing to the broader understanding of human experiences in immersive 
        environments.

        \subsection{Virtual Environement Design}

        The virtual design in this study has a practical role in helping participants gauge their speed while using the custom 
        go-kart. It includes recognizable features like roads, houses, and cars strategically placed within the virtual environment. 
        These act as reference points, making it easier for participants to perceive their speed as they navigate in the go-kart.

        The design is intentionally kept low-poly, using simplified geometric shapes. This choice ensures efficient processing, 
        contributing to a smooth and responsive experience – a crucial aspect for the entertainment-focused race simulator the 
        company aims to develop. The simplified visual style aligns with popular racing games, meeting the company's vision for an 
        accessible and enjoyable entertainment platform.

        In summary, the virtual design not only aids participants in the experiment by providing reference points for speed 
        perception but also fits the practical needs of the company looking to use it in a race simulator. The simplicity of the 
        design contributes to efficient graphics rendering, ensuring a seamless and engaging experience for users.
        

    \section{Discussion}
        \subsection{Implications for VR Design and Development}
        This study carries significant implications for the design and development of Virtual Reality (VR) experiences, offering 
        valuable insights that can shape the trajectory of the industry. The emphasis on incorporating clear points of reference 
        within the virtual environment to enhance speed perception underscores the importance of creating realistic and relatable 
        elements. VR designers can leverage this understanding to prioritize the inclusion of recognizable features, fostering a 
        heightened sense of realism and immersion in diverse virtual scenarios. This focus on enhancing user perception opens 
        avenues for more engaging and authentic VR experiences, transcending traditional boundaries.

        The study also sheds light on the importance of multi-sensory integration in VR design. 
        Understanding how participants perceive motion in both physical and virtual realms suggests that future developments can 
        explore ways to enhance user experiences by incorporating additional sensory cues. Haptic feedback, spatial audio, and 
        other immersive elements can be integrated to create a more comprehensive and engaging virtual environment, elevating the 
        overall user experience and expanding the possibilities of VR applications.

        The study's connection to a company interested in utilizing the virtual environment for entertainment purposes, particularly 
        in a race simulator, emphasizes the adaptability of VR technology across industries. VR developers can take inspiration from 
        this study to tailor their designs for entertainment applications, ensuring accessibility and enjoyment for a diverse user 
        base. This adaptability opens doors to innovative uses of VR beyond gaming, ranging from training simulations to educational 
        experiences and beyond.

        Furthermore, the insights gained from this study contribute to a deeper understanding of human-machine interaction, 
        particularly in scenarios involving motion perception. VR developers can leverage this understanding to refine applications 
        where user movements are a key element, such as driving simulations or virtual sports. By incorporating these implications 
        into future VR projects, designers and developers can collectively contribute to the evolution of VR technology, creating 
        more engaging, accessible, and versatile virtual experiences for users across various domains.

        \subsection{Integration with Exisiting Technology}

        The integration of the study's findings into existing technology holds promising opportunities across various domains. In the realm of entertainment and gaming, the low-poly design approach, aimed at efficient rendering and responsiveness, can be seamlessly incorporated into VR racing simulators. This integration enhances the realism and overall user experience, 
        providing racing enthusiasts with a more immersive and enjoyable virtual environment. Game developers can leverage these insights
        to refine existing racing games or create new ones that prioritize both visual fidelity and optimal processing, meeting the 
        demands of an evolving gaming landscape.

        Beyond entertainment, the study's implications have practical applications in training simulations. The insights into motion 
        perception can be particularly valuable for developing realistic driving and motion training programs. Integrating these 
        findings into existing training simulations for drivers, pilots, or individuals in dynamic professions can enhance the 
        effectiveness of these programs. By incorporating clear points of reference within the virtual environment, training 
        simulations can better simulate real-world scenarios, providing a more accurate and valuable learning experience for 
        participants.

        In the context of human-machine interaction, the study's understanding of motion perception can be integrated into the 
        development of innovative interfaces and control systems. For example, companies designing VR applications for training 
        operators of complex machinery or vehicles can utilize these insights to create interfaces that align more closely with 
        users' natural perceptions of motion, leading to improved usability and skill acquisition. Moreover, the study's adaptability
         to diverse applications suggests integration possibilities in fields such as healthcare, where VR is increasingly employed 
         for therapeutic interventions and medical training. By incorporating the principles of efficient rendering and realistic 
         motion perception, healthcare professionals can benefit from more immersive and effective VR tools, enhancing training 
         modules, procedural simulations, and patient treatments.
    
    \section{Conclusion}
        \subsection{Limitations of the Study}
    The study's insights into motion perception, while valuable, are constrained by several limitations. Notably, the exclusive focus 
    on linear motion, with participants navigating solely in a straight line within the custom go-kart, raises concerns about the 
    applicability of the findings to more complex driving scenarios involving turns and multidirectional movements. The study's emphasis
    on a swerve drivebase in the custom go-kart may limit its representativeness, as this system may not fully replicate the handling 
    characteristics of conventional vehicles.

    In conclusion, while offering valuable insights, the study's restrictions in linear motion, virtual environment scenarios, and 
    participant characteristics suggest that future research endeavors should consider more diverse motions, vehicles, and scenarios 
    for a comprehensive understanding of motion perception in both real and virtual driving 

\end{document}